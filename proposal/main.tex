\documentclass[a4paper,UKenglish,cleveref, autoref, thm-restate]{template/lipics-v2021}

\usepackage{float}
\usepackage{adjustbox}

\title{AFP Project: Implementing an efficient version of Data.Map in Agda}
\author{Wessel Custers}{Utrecht University, Netherlands}{}{}{}
\author{Sebastiaan Koppen}{Utrecht University, Netherlands}{}{}{}
\author{Daan van Westerlaak}{Utrecht University, Netherlands}{}{}{}

\authorrunning{W.R. Custers, S. Koppen, D. van Westerlaak}

\date{February 2025}

\bibliographystyle{acm}

\begin{document}

\maketitle

\section{Context}
Agda is a dependently typed programming language and proof assistant, which is mostly used for research purposes.
It has relatively few libraries, making it a less attractive language for development.
We believe that an implementation of the \texttt{Data.Map} module from Haskell's \texttt{containers} package would be a good addition for the language.

\texttt{Data.Map} is a finite map of key-value pairs. As can be read in the documentation, the implementation of \texttt{Data.Map} is based on size balanced trees \cite{adams1993functional, nievergelt1972binary}.

\section{Motivation}
None of the members of our team have experience with Agda. 
Based on what we do know about the language, we think this project should be a good opportunity to learn about Agda and what distinguishes it from Haskell.
Because we chose to work with a language none of us have experience with, we purposefully chose a topic that has clear bounds over a more creative one.
Similarly, because of the clear end result, we hope that the result of our project will be a library that is actually useful to others.

\section{Deliverables}
\begin{itemize}
    \item A pure-Agda implementation of \texttt{Data.Map}.
    \item Benchmarks for the implementation.
    \item \textit{(If possible):} Extensions on \texttt{Data.Map} (e.g. optimizations relating to underlying types or extra functions).
\end{itemize}

\section{Planning}
The total time for the project is approximately 6 - 7 weeks.

\begin{table}[h!]
    \centering
    \begin{tabular}{ r | c }
        Week & Content \\
        1 & Reading, understanding \texttt{Data.Map} \\
        2 &  \\
        3 &  \\
        4 &  \\
        5 &  \\
        6 &  \\
    \end{tabular}
    \caption{Schedule}
    \label{tab:schedule}
\end{table}


\bibliography{main}


\end{document}
