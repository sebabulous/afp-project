\documentclass[a4paper,UKenglish,cleveref, autoref, thm-restate]{template/lipics-v2021}

\usepackage{float}
\usepackage{adjustbox}

\title{AFP Project: Implementing an efficient version of Data.Map in Agda}
\author{Wessel Custers}{Utrecht University, Netherlands}{}{}{}
\author{Sebastiaan Koppen}{Utrecht University, Netherlands}{}{}{}
\author{Daan van Westerlaak}{Utrecht University, Netherlands}{}{}{}

\authorrunning{W.R. Custers, S. Koppen, D. van Westerlaak}

\date{February 2025}

\bibliographystyle{acm}

\begin{document}

\maketitle

\section{Context}
Agda is a dependently typed programming language and proof assistant, which is mostly used for research purposes. It has relatively few libraries, making it a less attractive language for development. We believe that an implementation of the \texttt{Data.Map} module from Haskell's \texttt{containers} package would be a good addition for the language.

\texttt{Data.Map} is a finite map of key-value pairs. As can be read in the documentation, the implementation of \texttt{Data.Map} is based on size balanced trees \cite{adams1993functional, nievergelt1972binary}.

\section{Motivation}
None of the members of our team have experience with Agda. Based on what we do know about the language, we think this project should be a good opportunity to learn about Agda and what distinguishes it from Haskell. We purposefully chose a topic that has clear bounds instead of one that might be considered more creative. This should allow us to deliver a high-quality package that might actually be useful to others.

\section{Deliverables}
\begin{itemize}
    \item A pure-Agda implementation of \texttt{Data.Map}.
    \item Benchmarks for the implementation.
    \item \textit{(If possible):} Extensions on \texttt{Data.Map} (e.g. optimizations relating to underlying types or extra functions).
\end{itemize}

\section{Planning}
The total time for the project is approximately 6 - 7 weeks. Our work can be split into 3 phases. First, we want to have a good understanding of \texttt{Data.Map} and its underlying theories (1-2 weeks). Using this understanding, we will then make a translation to a pure Agda implementation while looking for possible optimizations that are not possible in Haskell (2-3 weeks). Finally, we will write benchmarks and, where applicable, proofs for our new implementation (2-3 weeks).


\bibliography{main}


\end{document}
